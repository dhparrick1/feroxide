\documentclass[11pt]{article}

\usepackage{bera}
\def\code#1{\texttt{#1}}

\title{String representation}
\author{The feroxide team}

\begin{document}
\maketitle

\section{Introduction}
Feroxide works by nesting all kinds of elements. This is ideal for calculation
purposes, but it can be a pain if you have to type it all out every time again.
That is what the string representation was made for. With that, you are able
to digitalise complex reactions in only a single line of code, instead of
using tens of lines of code to achieve the same thing.

The downside of using string representation is that it is a bit slower at
runtime. That is also a reason why you should preferably only initialise
a reaction once, instead of having it recreated in a loop. \\

This paper is made to explain how the string representation of all
elements should be created and parsed. It is described from the smallest
parts to the biggest parts. Every bigger part depends on the smaller parts.

\newpage

\section{\code{Atom}}
Atoms are as easy to represent as it can get. Simply enter the atom symbol to refer to the corresponding Atom.

\subsection{Examples}
\begin{itemize}
  \item{\code{C} will result in Carbon}
  \item{\code{Co} will result in Cobalt}
  \item{\code{H} will result in Hydrogen}
\end{itemize}


\section{\code{MoleculeCompound}}
MoleculeCompounds are basically atoms with an amount. This is entered by
putting a number after the atom symbol. If a number is omitted, one (1) is assumed.

\subsection{Examples}
\begin{itemize}
  \item{\code{H2} will result in $H_2$}
  \item{\code{C3} will result in $C_3$. This may be used to form butane}
  \item{\code{S} will result in $S_1$, or simply $S$}
\end{itemize}


\section{\code{Molecule}}
Molecules are a combination of one or more MoleculeCompounds. It starts with
an amount, which represents the amount of moles. If no amount is given,
one (1) is assumed.

\subsection{Examples}
\begin{itemize}
  \item{\code{H2O} will result in $1H_2O_1$, or simply $H_2O$}
  \item{\code{5CH4} will result in $5C_1H_4$, or simply $5CH_4$}
  \item{\code{H3C6H5O7} will result in $H_3C_6S_5$}
\end{itemize}

\subsection{Bugs}
\begin{itemize}
\item{Currently you can not add water molecules to the crystal structure of a molecule. For example, the following is \textbf{invalid}: \code{CuSO4.5H2O}}
\item{Molecule structures are not yet implemented. This means isomers such as $CH_3CH_2CH_2OH$ (propan-1-ol) and $CH_3CHOHCH_3$ (propan-2-ol) are the same to feroxide, namely \code{C3H8O}.}
\end{itemize}


\section{\code{Ion}}
Ions are molecules with charge. This charge is entered after the molecule, seperated with a semi-colon (\code{;}). Negative charges should have a dash (\code{-}) \textbf{after} the charge. Positive charges should \textbf{not} have a \code{+}, but only the number.

\subsection{Special cases}
\begin{itemize}
  \item{If no semi-colon and thus no charge is given, a charge of zero (0) is assumed.}
  \item{If only a semi-colon is given, a charge of one (1) is assumed.}
  \item{If only a semi-colon and a dash are given, a charge of negative one (-1) is assumed.}
  \item{If the "atom" is an electron (\code{e}), then a charge of negative one (-1) is assumed. Any given charge will be ignored.}
\end{itemize}

\subsection{Examples}
\begin{itemize}
  \item{\code{H2} will result in $H_2^0$, or simply $H_2$}
  \item{\code{Fe;3} will result in $Fe^{3+}$}
  \item{\code{SO4;2-} will result in $S_1O_4^{2-}$, or simply $SO_4^{2-}$}
  \item{\code{OH;-} will result in $O_1H_1^{1-}$, or simply $OH^{-}$}
  \item{\code{e} will result in $e_1^{1-}$, or simply $e^-$}
  \item{\code{e;5} will \textbf{also} result in $e^-$}
\end{itemize}

\section{ReactionCompound}

\section{ReactionSide}

\section{Reaction}


\end{document}
